\documentclass[a4paper, titlepage]{article}
\usepackage{HYY_style}
% \usepackage{setspace, changepage, enumitem, verbatim}
% \usepackage{tikz, fbox, cancel}
\usepackage[utf8]{inputenc}
\usepackage[T1]{fontenc}
\usepackage{lmodern}
\usepackage{epstopdf}
\usepackage{matlab}
\usepackage{enumitem}
\newcommand{\tabincell}[2]{\begin{tabular}{@{}#1@{}}#2\end{tabular}}
\allowdisplaybreaks[4]
\begin{document}
    \renewcommand{\thefootnote}{\fnsymbol{footnote}}
    \title{{\small《MATLAB程序设计》大作业个人汇报}\\COVID-19疫情数据的简要统计分析与预测\footnotemark[1]}
    \author{王逸扬(19300180016)\\杨耕智(19300180112){\kaishu }}
    \date{2022年5月21日}
    \footnotetext[1]{\kaishu 相关代码及其版本演进托管在\url{https://github.com/maix00/StatisticsAnalysis}.}
    \maketitle
    \renewcommand{\thefootnote}{\araboc{footnote}}
    \renewcommand{\contentsname}{\centering 目录}
    \tableofcontents

    \section{整体分析思路与代码分工}
          众所周知,2019年底开始的COVID-19疫情给人们的生产生活造成了极大的影响,如何利用疫情相关数据、如何通过数据挖掘、分析、预测,以指导人们作出更加有利于社会发展的决策,是当今较为热门的研究方向. 在此,我们将根据所给的有关数据,尽可能地贴合这个目的,做出一些简要的分析与预测.

        \subsection{数据观察与简要分析}
          首先,观察所给数据. 有两个数据集,一个是\texttt{country.csv}(以下称「国家数据」),另一个是\texttt{daily\_info.csv}(以下称「每日数据」). 每日数据中包括$21$个国家的有关数据,其中Senegal在国家数据中有数据缺失,遂只对余下$20$个国家进行分析,分别是Australia, Canada, Chile, China, Egypt, Ethiopia, France, Germany, Hungary, Iceland, Japan, Malaysia, Saudi Arabia, South Africa, South Korea, Sweden, Thailand, United States, 以及Zimbabwe. 

          这$20$个国家的国家数据中,每个国家都有$12$个特征,简要分类:
        {\kaishu
        \begin{itemize}[itemsep=-1pt,topsep=1pt]
            \item [\textbf{地理位置}:]所属大洲\texttt{continent};
            \item [\textbf{人口成分}:]人口\texttt{population}, 每平方千米人口密度\texttt{population\_density}, 中位年龄\texttt{median\_age}, $65$岁以上人口占比\texttt{aged\_65\_older}, $70$岁以上人口占比\texttt{aged\_70\_older};
            \item [\textbf{经济环境}:]按购买力平价的人均国民生产总值\texttt{gdp\_per\_capita};
            \item [\textbf{卫生水平}:]男性吸烟率\texttt{male\_smokers}, 女性吸烟率\texttt{female\_smokers}, 每年每十万人心血管疾病死亡率\texttt{cardiovasc\_death\_rate}, 每千人床位数\texttt{hospital\_beds\_per\_thousands}, 预期寿命\texttt{life\_} \texttt{expectancy}.
        \end{itemize}
        }
        据此,我们可以对这些国家进行无监督聚类分析,在每一类中选取一个代表性的国家构建一分析预测模型(利用每日数据和/或国家特征),并对同一类中的其他国家验证该模型的有效性. 但是,需要指出,因为所给数据的有限性,所建立或选择的模型是有限的,这里的特征作为模型参数的个数也是有限的,因此这里的多数特征在本文中只能用于聚类.

          对于每日数据,我们首先指出,在数据的导入与缺失值处理上,这个数据集遇到了一些困难,对于这些问题,我们将在节\ref{数据的导入}与节\ref{数据的缺失值处理}中分别尝试解决. 下面,对每日数据的各字段(除所属大洲与时间戳外),简要分类:
        {\kaishu
        \begin{itemize}[itemsep=-1pt,topsep=1pt]
            \item [\textbf{严重程度}:] 新增与累计病例:\texttt{new\_cases}, \texttt{total\_cases}; 住院与重症监护ICU人数:\texttt{hosp\_patients}, \texttt{icu\_} \texttt{patients}; 近七日平均检测阳性率:\texttt{positive\_rate};
            \item [\textbf{检测能力}:] 新增与累计检测人次:\texttt{new\_tests}, \texttt{total\_tests}; 近七日平均检测阳性率:\texttt{positive\_rate};
            \item [\textbf{疫苗接种}:] 新增与累计疫苗接种人次:\texttt{new\_vaccinations}, \texttt{total\_vaccinations}; 
            \item [\textbf{管控力度}:] 政府管控力度:\texttt{stringency\_index}.
        \end{itemize}
        }
        这些时间序列数据,(1)可以用来构建一分析预测模型,(2)也可以取某一时间段内的最大、平均等统计指标(时间序列特征),用来对不同国家进行监督或无监督聚类分析. 这里的聚类分析,得到的是不同国家的不同疫情发展模式,与前面的利用国家数据得到的关于国家特性的聚类不同.

          另外,需要指出,因为所给数据的有限性,所能建立或选择的分析预测模型是极其有限的,比如,由于缺少治愈、死亡数据,不能采用传统的传染病模型SIR至SEIR等,又比如,由于缺少流调数据,无法统计计算基本再生数$R_0$、潜伏期平均长度$\bar{T_L}$等与传染病有关的统计指标. 此外,正因为所能建立或选择的模型是有限的,每日数据中的一些字段在本文中可能不会被用到.

        \subsection{整体分析思路}
          综合上面的简要分析,我们将本文的整体分析思路概括如下.

        \paragraph{\fbox{第一种路径}} 根据国家特征进行聚类,而后利用代表国家的每日数据构造一分析预测模型.
        \begin{itemize}[itemsep=-1pt,topsep=1pt]
            \item [\textbf{第一步}:](聚类分析)对$20$个国家的特征数据,先进行主成分分析,选取前$5$个主特征后进行$5$-mean聚类分析.
            \item [\textbf{第二步}:](构建模型)基于LSTM神经网络模型,对代表国家一定时期内的每日新增病例数据进行学习.
            \item [\textbf{第三步}:](预测检验)利用上一步得到的模型,对时期外的数据进行预测检验,并对同一类中的其他国家的数据进行预测检验.
        \end{itemize}
        此路径能得到一种基于LSTM神经网络的分析预测模型.

        \paragraph{\fbox{第二种路径}} 根据每日数据和/或国家数据构造时间序列特征,而后进行谱系聚类.
        \begin{itemize}[itemsep=-1pt,topsep=1pt]
            \item [\textbf{第一步}:](构造特征)
            \item [\textbf{第二步}:](聚类分析)
        \end{itemize}
        此路径能对不同国家的疫情发展模式进行分类,针对分类结果可以提出更有针对性的抑制疫情发展的建议. \url{https://www.it610.com/article/1515470890224123904.htm}.
        
        \paragraph{\fbox{第三种路径}} 拟合部分国家疫情发展初期缺失的数据.

        \url{https://www.it610.com/article/1515470890224123904.htm}.

        \subsection{代码分工}
          因为疫情的关系,我们两位同学,不得不线上联系,我们也因此第一次尝试使用github平台. 仓库网址为\url{https://github.com/maix00/StatisticsAnalysis}. 有关代码分工的详细情况,在平台上可以清晰看到,这里就作简单介绍.

          节\ref{数据的导入}与节\ref{数据的缺失值处理}数据导入、缺失值处理的有关代码主要是由王逸扬同学完成的,节\ref{数据分析}数据分析的有关代码主要是由杨耕智同学完成的.

    \section{数据的导入与检索}\label{数据的导入}
        导入数据已有可以使用的\texttt{readtable}函数,且可以通过\texttt{detectImportOptions}函数,在导入数据前,预先探测并修改导入参数. 在此过程中,我们发现如下问题:
        \begin{enumerate}
            \item [(1)] 以每日数据的\texttt{new\_vaccinations}与\texttt{total\_vaccinations}变量为例. 由于疫情发生初期长时间没有此类数据,\texttt{detectImportOptions}函数探测认为,此变量的变量类型是为\texttt{char},因此需要手动将导入参数修改为\texttt{double}.
            \item [(2)] 在导入每日数据时,通常最后需要检索出某个国家、某个时间段的数据,目的是形成时序数据. 但是,为此,MATLAB可以通过索引检索,但是不同类型的数据检索的方式不同,不是很便利,没有专门的检索函数可以使用.
        \end{enumerate}
        为了解决上面发现的问题,并寄希望于能在单一函数中解决全部的导入与检索问题,我们利用了MATLAB的面向对象编程,编写了一个类\texttt{StatisticsAnalysis}(之所以取这个名字,是因为之后还将赋予它更多的功能). 下面用一个例子来阐述这个类的作用. 比如,我们想要导入法国在2020年的新增与累计接种疫苗人次——
\begin{matlabcode}
daily_SA = StatisticsAnalysis( ...
    'TablePath', './data/COVID19/daily_info.csv', ...
    'ImportOptions', { ...
        'VariableTypes', { ...
            'new_vaccinations', 'double',
            'total_vaccinations', 'double' ...
            }, ...
        'SelectedVariableNames', ...
            {'date', 'new_vaccinations', 'total_vaccinations'} ...
        }, ...
    'SelectTableOptions', { ...
        'location', 'France', ...
        'date', timerange("2020-01-01", "2020-12-31", 'closed') ...
        } ...
    );
daily = daily_SA.TimeTable;
\end{matlabcode}
        其中,\texttt{"TablePath"}、\texttt{"ImportOptions"}和\texttt{"SelectTableOptions"}是函数参数名,\texttt{TimeTable}是该类的一个属性. 如果我们还想导入法国在2020年的新增与累计病例数,我们只需——
\begin{matlabcode}
daily2 = daily_SA.Update('ImportOptions', { ...
        'SelectedVariableNames', {'date', 'new_cases', 'total_cases'} ...
        }...
    ).TimeTable;
\end{matlabcode}
        上面的方法不会重复导入、检索数据. 更多关于这个类的信息,请参见附录\ref{app:StatisticsAnalysis}. 
    \section{数据的缺失值处理}\label{数据的缺失值处理}
        以每日数据为例,我们发现如下问题:
        \begin{enumerate}
            \item [(1)] 如果要分析变量\texttt{new\_vaccinations}或\texttt{total\_vaccinations},那么在疫情初期,有很多缺失值,可能需要填充为$0$.
            \item [(2)] 变量\texttt{new\_cases}可能会出现意外的缺失值,但能从\texttt{total\_cases}变量中恢复.
            \item [(3)] 一些国家的数据,可能由于更正数据的需要,部分日期的\texttt{total\_cases}会比上一日的少,以致当日的\texttt{new\_cases}成为缺失值.
            \item [(4)] 一些不能恢复的变量,可能需要线性插值,或者其他的方法连接.
            \item [(5)] 在统计上不可避免出现一些随机波动,可能需要光滑处理.
        \end{enumerate}
        为了解决这些问题,我们编写了一个函数\texttt{tableMissingValuesHelper.m},可以用来分析缺失值的分布,在这之后再经过人工判断,可以选择不同的参数,以恢复数据. 更多关于这个函数的信息,请参见附录\ref{app:tableMissingValuesHelper}. 
    \section{数据分析过程}\label{数据分析}
    \subsection{一般的统计指标}\label{一般的统计指标}
    \subsection{聚类分析}
    \subsection{疫情数据的分析预测}
    \newpage
    \appendix
    \section{\texttt{StatisticsAnalysis}类}\label{app:StatisticsAnalysis}
      本类存储在\texttt{./functions/@StatisticsAnalysis}.本类目前有两个功能,一是数据的导入与检索,二是对表格进行初步的统计分析,即计算传入的或预设的统计指标. 

      \texttt{StatisticsAnalysis}生成函数,主要的参数包括\texttt{ImportOptions},\texttt{SelectTableOptions}和\texttt{TagsGenerateOptions}. 其实现的主要过程见下.
\begin{matlabcode}
TablePath作为参数被传入时——
Step 1: detectImportOptions
Step 2: 导入与检索表格
    Step (1): 用ImportOptions,修改detectImportOptions,并导入表格
    Step (2): 引用./function/selecttable.m函数检索表格
    Step (3): 去除多余的列
Step 3: 添加变量标签Tag
        并将标签对应的统计指标添加到Table.Properties.CustomProperties
    Step (1): 引用类方法TagsGenerate以生成标签、计算统计指标
    Step (2): 引用addprop函数,将统计指标添加到..CustomProperties
\end{matlabcode}
\begin{matlabcode}
Table作为参数被传入时——
Step 1: 检索表格
    if SelectTableOptions非空
        引用./function/selecttable.m函数检索表格
    end
Step 2: 添加变量标签Tag
        并将标签对应的统计指标添加到Table.Properties.CustomProperties
    Step (1): 引用类方法TagsGenerate以生成标签、计算统计指标
    Step (2): 引用addprop函数,将统计指标添加到..CustomProperties
\end{matlabcode}
    其中,各种\texttt{Options}参数,可以是\texttt{struct}类型,也可以是先选项名、后选项值的\texttt{cell}类型. 可以参见节\ref{数据的导入}中的例子.

      \texttt{ImportOptions}所接受的选项名与选项值,请参考\texttt{detectImportOptions}中列明的属性名与属性值.

      \texttt{SelectTableOptions}所接受的选项名,须是表格的列名,所接受的选项值,可以是单一的值,可以是\texttt{timerange}类型,也可以是\texttt{./functions/@arange}.

      前两种\texttt{Options}中,多个选项值可以\texttt{array}或\texttt{1x1 cell}的形式传入,不同的选项值将分别用于表格的导入、检索,最后尽可能地上下拼合表格.

      \texttt{TagsGenerateOptions}所接受的选项名,主要的即\texttt{CustomTagName}和\texttt{CustomTagFunction},前者给表格的列赋予标签、后者给不同的标签赋予统计指标(函数句柄). 预设的变量标签、预设的统计指标,以及参数的格式,可以用\texttt{help}查看.

      类方法中比较实用的有\texttt{TagsGenerate}与\texttt{Update},前者用作统计分析,后者用来更新各种\texttt{Options}参数.

    \section{\texttt{selecttable}函数}\label{app:selecttable}
    本函数存储在\texttt{./functions}.
    \section{\texttt{tableMissingValuesHelper}函数}\label{app:tableMissingValuesHelper}
    本函数存储在\texttt{./functions}.
    \section{利用有关数据可以进行的可视化}
\end{document}