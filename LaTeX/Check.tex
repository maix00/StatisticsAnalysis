\documentclass[a4paper, titlepage]{article}
\usepackage{HYY_style}
% \usepackage{setspace, changepage, enumitem, verbatim}
% \usepackage{tikz, fbox, cancel}
\usepackage[utf8]{inputenc}
\usepackage[T1]{fontenc}
\usepackage{lmodern}
\usepackage{epstopdf}
\usepackage{matlab}
\newcommand{\tabincell}[2]{\begin{tabular}{@{}#1@{}}#2\end{tabular}}
\allowdisplaybreaks[4]
\begin{document}
    \renewcommand{\thefootnote}{\fnsymbol{footnote}}
    \title{{\small《MATLAB程序设计》大作业个人汇报}\\COVID-19疫情数据的统计分析\footnotemark[1]}
    \author{王逸扬(19300180016)\\杨耕智(19300180112){\kaishu }}
    \date{2022年5月21日}
    \footnotetext[1]{\kaishu 相关代码及其版本演进托管在\url{https://github.com/maix00/StatisticsAnalysis}.}
    \maketitle
    \renewcommand{\thefootnote}{\araboc{footnote}}
    \renewcommand{\contentsname}{\centering 目录}
    \tableofcontents

    \section{整体分析思路}
    \section{数据的导入}
        导入数据已有可以使用的\texttt{readtable}函数,且可以通过\texttt{detectImportOptions}函数,在导入数据前,预先探测并修改导入参数. 在此过程中,我们发现如下问题:
        \begin{enumerate}
            \item [(1)] 以\texttt{daily\_info.csv}数据集的\texttt{new\_vaccinations}变量为例. 由于疫情发生初期长时间没有此类数据,\texttt{detectImportOptions}函数探测认为,此变量的变量类型是为\texttt{char},因此需要手动修改为\texttt{double}.
            \item [(2)] 在导入\texttt{daily\_info.csv}数据集时,通常最后需要检索出某个国家、某个时间段的数据. 为此,MATLAB没有专门的检索函数可以使用.
            \item [(3)] 当数据集特别大,而我们只需要个别列、以及检索出的个别行的数据时,需要手动设置导入选项以节省存储空间.
        \end{enumerate}
        为了解决上面发现的问题,并寄希望于能在单一函数中解决全部的导入问题,我们利用了MATLAB的面向对象编程. 下面用一个例子来阐述这个类的作用. 比如,我们想要导入法国在2020年的全体新增接种疫苗数与总接种疫苗数的数据——
\begin{matlabcode}
path_daily = './data/COVID19/daily_info.csv';
data = StatisticsAnalysis( ...
    'TablePath', path_daily, ...
    'ImportOptions', { ...
        'VariableTypes', { ...
            'new_vaccinations', 'double';
            'total_vaccinations', 'double' ...
            }; ...
        'SelectedVariableNames', ...
            {'date', 'new_vaccinations', 'total_vaccinations'} ...
        }, ...
    'SelectTableOptions', { ...
        'location', 'France'; ... Country
        'date', timerange("2020-01-01", "2020-12-31", 'closed') ...
    } ... % Select Table Before Importing, Please Set 'SelectFirst' true
    ).Table;
\end{matlabcode}
        \begin{sloppypar}
            \flushleft
            其中,\texttt{StatisticsAnalysis}是该类的生成函数,\texttt{"TablePath"}、\texttt{"ImportOptions"}和\texttt{"SelectTab} \texttt{leOptions"}是函数参数名,\texttt{.Table}是该类的一个属性. 之所以取\texttt{StatisticsAnalysis}这个名字,是因为后面将赋予它更多的功能. 更多关于这个类的信息,请参见附录\ref{app:StatisticsAnalysis}. %这个类能够在导入全部所需数据前,先仅导入所要检索的变量所在列的数据,检索出符合条件的行,转化为\texttt{detectImportOptions}所接受的\texttt{DataLines}属性,再合并上\texttt{"ImportOptions"}的函数参量以导入数据.
        \end{sloppypar}
    \section{数据的缺失值处理}
        我们发现,无论是\texttt{}
    \section{数据分析过程}
    \subsection{一般的统计指标}
    \subsection{聚类分析}
    \subsection{疫情数据的分析预测}
    \newpage
    \appendix
    \section{\texttt{StatisticsAnalysis}类}\label{app:StatisticsAnalysis}
      本类的主体与相关方法存储在\texttt{./functions/@StatisticsAnalysis}.本类目前只有两个功能,一是数据的导入与检索,二是对表格进行初步的统计分析,计算传入的或预设的统计指标. 
    
      \texttt{StatisticsAnalysis}生成函数,其实现的伪代码见下.
\begin{matlabcode}
TablePath作为参数被传入时——
Step 1: detectImportOptions
Step 2: 导入与检索表格
    if SelectFirst && SelectTableOptions非空
        % 节省内存的导入方法,但额外耗费了一定时间
        Step (1): 导入表格,列只包含需要检索的变量
        Step (2): 引用./function/selecttable.m函数检索表格,获取剩余的行数
        Step (3): 将刚获取的剩余行数,更新到ImportOptions
        Step (4): 将ImportOptions作为参数,导入表格
    elseif ~SelectFirst && SelectTableOptions非空
        Step (1): 将需要检索的变量,作为需要导入的列,更新到ImportOptions
        Step (2): 将ImportOptions作为参数,导入表格
        Step (3): 引用./function/selecttable.m函数检索表格
        Step (4): 去除多余的列(前面辅助检索的列)
    end
Step 3: 添加变量标签Tag
        并将标签对应的统计指标添加到Table.Properties.CustomProperties
    Step (1): 引用类方法TagsGenerate以生成标签、计算统计指标
    Step (2): 引用addprop函数,将统计指标添加到..CustomProperties
\end{matlabcode}
\begin{matlabcode}
Table作为参数被传入时——
Step 1: 检索表格
    if SelectTableOptions非空
        引用./function/selecttable.m函数检索表格
    end
Step 2: 添加变量标签Tag
        并将标签对应的统计指标添加到Table.Properties.CustomProperties
    Step (1): 引用类方法TagsGenerate以生成标签、计算统计指标
    Step (2): 引用addprop函数,将统计指标添加到..CustomProperties
\end{matlabcode}
    \texttt{StatisticsAnalysis}生成函数,其所有的传入参数包括:\texttt{Table}, \texttt{TablePath}, \texttt{ImportOptions}, \texttt{Select} \texttt{TableOptions}, \texttt{SelectFirst}, \texttt{TagsGenerate}, \texttt{TagsGenerateOptions}. 其中,各种\texttt{Options}参数,要求是左侧为选项名、右侧是选项值的\texttt{cell}.

      \texttt{TagsGenerateOptions}所接受的选项名,或\texttt{TagsGenerate}方法所接受的传入参数名,包括:\texttt{TagContinuity}, \texttt{TagCategory}, \texttt{CustomTagName}, \texttt{CustomTagFunction}\!等. 预设的变量标签包括:\texttt{unique}, \texttt{invariant}, \texttt{logical}, \texttt{categorical}, \texttt{discrete}, \texttt{continuous}.

    \section{\texttt{selecttable}函数}\label{app:selecttable}
    本函数存储在\texttt{./functions}
\end{document}