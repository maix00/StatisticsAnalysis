\documentclass[a4paper, titlepage]{article}
%\usepackage{HYY_style}
\usepackage{amsmath,amsfonts,amssymb,amsthm,bm,extarrows,mathrsfs,dutchcal}
\usepackage{ctex}%中文支持
\usepackage{geometry}%页面布局
	\geometry{top=25mm,bottom=25mm,left=30mm,right=30mm}%页面布局
	\setlength\parindent{2em}% 每行缩进两个汉字
	\setmainfont{Times New Roman}% 设置字体
	%\setmonofont{Courier New}
	\setsansfont{Arial}
	\setCJKfamilyfont{kai}[AutoFakeBold]{simkai.ttf}
	\newcommand*{\kai}{\CJKfamily{kai}}
	\setCJKfamilyfont{song}[AutoFakeBold]{SimSun}
	\newcommand*{\song}{\CJKfamily{song}}
\usepackage{xcolor}%颜色
\usepackage{graphicx}%图片
\usepackage{setspace}
% \usepackage{setspace, changepage, enumitem, verbatim}
% \usepackage{tikz, fbox, cancel}
\usepackage[utf8]{inputenc}
\usepackage[T1]{fontenc}
\usepackage{lmodern}
\usepackage{epstopdf}
\usepackage{matlab}
\usepackage{enumitem}
\usepackage{longtable, booktabs, multirow, makecell, threeparttable, ltxtable, supertabular, array}
\newcommand{\tabincell}[2]{\begin{tabular}{@{}#1@{}}#2\end{tabular}}
\allowdisplaybreaks[4]
\begin{document}
    \renewcommand{\thefootnote}{\fnsymbol{footnote}}
    \title{{\small《MATLAB程序设计》大作业个人汇报}\\COVID-19疫情数据的简要统计分析与预测\footnotemark[1]}
    \author{王逸扬(19300180016)\\杨耕智(19300180112){\kaishu }}
    \date{2022年5月21日}
    \footnotetext[1]{\kaishu 相关代码及其版本演进托管在\url{https://github.com/maix00/StatisticsAnalysis}.}
    \maketitle
    \renewcommand{\thefootnote}{\araboc{footnote}}
    \renewcommand{\contentsname}{\centering 目录}
    \tableofcontents

    \section{整体分析思路与代码分工}
          众所周知,2019年底开始的COVID-19疫情给人们的生产生活造成了极大的影响,如何利用疫情相关数据、如何通过数据挖掘、分析、预测,以指导人们作出更加有利于社会发展的决策,是当今较为热门的研究方向. 在此,我们将根据所给的有关数据,尽可能地贴合这个目的,做出一些简要的分析与预测.

        \subsection{数据观察与简要分析}
          首先,观察所给数据. 有两个数据集,一个是\texttt{country.csv}(以下称「国家数据」),另一个是\texttt{daily\_info.csv}(以下称「每日数据」). 每日数据中包括$21$个国家的有关数据,其中Senegal在国家数据中有数据缺失,遂只对余下$20$个国家进行分析,分别是Australia, Canada, Chile, China, Egypt, Ethiopia, France, Germany, Hungary, Iceland, Japan, Malaysia, Saudi Arabia, South Africa, South Korea, Sweden, Thailand, United States, 以及Zimbabwe. 

          这$20$个国家的国家数据中,每个国家都有$12$个特征,简要分类:
        {\kaishu
        \begin{itemize}[itemsep=-1pt,topsep=1pt]
            \item [\textbf{地理位置}:]所属大洲\texttt{continent};
            \item [\textbf{人口成分}:]人口\texttt{population}, 每平方千米人口密度\texttt{population\_density}, 中位年龄\texttt{median\_age}, $65$岁以上人口占比\texttt{aged\_65\_older}, $70$岁以上人口占比\texttt{aged\_70\_older};
            \item [\textbf{经济环境}:]按购买力平价的人均国民生产总值\texttt{gdp\_per\_capita};
            \item [\textbf{卫生水平}:]男性吸烟率\texttt{male\_smokers}, 女性吸烟率\texttt{female\_smokers}, 每年每十万人心血管疾病死亡率\texttt{cardiovasc\_death\_rate}, 每千人床位数\texttt{hospital\_beds\_per\_thousands}, 预期寿命\texttt{life\_} \texttt{expectancy}.
        \end{itemize}
        }
        据此,我们可以对这些国家进行无监督聚类分析,在每一类中选取一个代表性的国家构建一分析预测模型(利用每日数据和/或国家特征),并对同一类中的其他国家验证该模型的有效性. 但是,需要指出,因为所给数据的有限性,所建立或选择的模型是有限的,这里的特征作为模型参数的个数也是有限的,因此这里的多数特征在本文中只能用于聚类.

          对于每日数据,我们首先指出,在数据的导入与缺失值处理上,这个数据集遇到了一些困难,对于这些问题,我们将在节\ref{数据的导入}与节\ref{数据的缺失值处理}中分别尝试解决. 下面,对每日数据的各字段(除所属大洲与时间戳外),简要分类:
        {\kaishu
        \begin{itemize}[itemsep=-1pt,topsep=1pt]
            \item [\textbf{严重程度}:] 新增与累计病例:\texttt{new\_cases}, \texttt{total\_cases}; 住院与重症监护ICU人数:\texttt{hosp\_patients}, \texttt{icu\_patients}; 近七日平均检测阳性率:\texttt{positive\_rate};
            \item [\textbf{检测能力}:] 新增与累计检测人次:\texttt{new\_tests}, \texttt{total\_tests}; 近七日平均检测阳性率:\texttt{positive\_rate};
            \item [\textbf{疫苗接种}:] 新增与累计疫苗接种人次:\texttt{new\_vaccinations}, \texttt{total\_vaccinations}; 
            \item [\textbf{管控力度}:] 政府管控力度:\texttt{stringency\_index}.
        \end{itemize}
        }
        这些时间序列数据,(1)可以用来构建一分析预测模型,(2)也可以取某一时间段内的最大、平均等统计指标(时间序列特征),用来对不同国家进行监督或无监督聚类分析. 这里的聚类分析,得到的是不同国家的不同疫情发展模式,与前面的利用国家数据得到的关于国家特性的聚类不同.

          另外,需要指出,因为所给数据的有限性,所能建立或选择的分析预测模型是极其有限的,比如,由于缺少治愈、死亡数据,不能采用传统的传染病模型SIR至SEIR等,又比如,由于缺少流调数据,无法统计计算基本再生数$R_0$、潜伏期平均长度$\bar{T_L}$等与传染病有关的统计指标. 此外,正因为所能建立或选择的模型是有限的,每日数据中的一些字段在本文中可能不会被用到.

        \subsection{整体分析思路}
          综合上面的简要分析,我们将本文的整体分析思路概括如下.

        \paragraph{\fbox{第一种路径}} 根据国家特征进行聚类,而后利用代表国家的每日数据构造一分析预测模型.
        \begin{itemize}[itemsep=-1pt,topsep=1pt]
            \item [\textbf{第一步}:](聚类分析)对$20$个国家的特征数据,先进行主成分分析,选取前$5$个主特征后进行$5$-mean聚类分析.
            \item [\textbf{第二步}:](构建模型)基于LSTM神经网络模型,对代表国家一定时期内的每日新增病例数据进行学习.
            \item [\textbf{第三步}:](预测检验)利用上一步得到的模型,对时期外的数据进行预测检验,并对同一类中的其他国家的数据进行预测检验.
        \end{itemize}
        此路径能得到一种基于LSTM神经网络的分析预测模型.

        \paragraph{\fbox{第二种路径}} 根据每日数据和/或国家数据构造时间序列特征,而后进行谱系聚类.
        \begin{itemize}[itemsep=-1pt,topsep=1pt]
            \item [\textbf{第一步}:](构造特征)
            \item [\textbf{第二步}:](聚类分析)
        \end{itemize}
        此路径能对不同国家的疫情发展模式进行分类,针对分类结果可以提出更有针对性的抑制疫情发展的建议. \url{https://www.it610.com/article/1515470890224123904.htm}.
        
        \paragraph{\fbox{第三种路径}} 拟合部分国家疫情发展初期缺失的数据.

        \url{https://www.it610.com/article/1515470890224123904.htm}.

        \subsection{代码分工}
          因为疫情的关系,我们两位同学,不得不线上联系,我们也因此第一次尝试使用github平台. 仓库网址为\url{https://github.com/maix00/StatisticsAnalysis}. 有关代码分工的详细情况,在平台上可以清晰看到,这里就作简单介绍.

          节\ref{数据的导入}与节\ref{数据的缺失值处理}数据导入、缺失值处理的有关代码主要是由王逸扬同学完成的,节\ref{数据分析}数据分析的有关代码主要是由杨耕智同学完成的.

    \section{数据的导入与检索}\label{数据的导入}
        导入数据已有可以使用的\texttt{readtable}函数,且可以通过\texttt{detectImportOptions}函数,在导入数据前,预先探测并修改导入参数. 在此过程中,我们发现如下问题:
        \begin{enumerate}
            \item [1.] 以每日数据的\texttt{new\_vaccinations}与\texttt{total\_vaccinations}变量为例. 由于疫情发生初期长时间没有此类数据,\texttt{detectImportOptions}函数探测认为,此变量的变量类型是为\texttt{char},因此需要手动将导入参数修改为\texttt{double}.
            \item [2.] 在导入每日数据时,通常最后需要检索出某个国家、某个时间段的数据,目的是形成时序数据. 为此,MATLAB可以通过括号索引检索,但是不同类型的数据检索的方式不同,不是很便利,没有专门的检索函数可以使用.
        \end{enumerate}
        为了解决上面发现的问题,并寄希望于能在单一函数中解决全部的导入与检索问题,我们利用了MATLAB的面向对象编程,编写了一个类\texttt{StatisticsAnalysis}(之所以取这个名字,是因为之后还将赋予它更多的功能). 下面用一个例子来阐述这个类的作用. 比如,我们想要导入法国在2020年的新增与累计接种疫苗人次——
\begin{matlabcode}
daily_SA = StatisticsAnalysis( ...
    'TablePath', './data/COVID19/daily_info.csv', ...
    'ImportOptions', { ...
        'VariableTypes', { ...
            'new_vaccinations', 'double', ...
            'total_vaccinations', 'double' ...
            }, ...
        'SelectedVariableNames', ...
            {'date', 'new_vaccinations', 'total_vaccinations'} ...
        }, ...
    'SelectTableOptions', { ...
        'location', 'France', ...
        'date', timerange("2020-01-01", "2020-12-31", 'closed') ...
        } ...
    );
daily = daily_SA.TimeTable;
\end{matlabcode}
        其中,\texttt{"TablePath"}、\texttt{"ImportOptions"}和\texttt{"SelectTableOptions"}是函数参数名,\texttt{TimeTable}是该类的一个属性. 如果我们还想导入法国在2020年的新增与累计病例数,我们只需——
\begin{matlabcode}
daily2 = daily_SA.Update('ImportOptions', { ...
        'SelectedVariableNames', {'date', 'new_cases', 'total_cases'} ...
        }...
    ).TimeTable;
\end{matlabcode}
        上面的方法不会重复导入、检索数据. 更多关于这个类的信息,请参见附录\ref{app:StatisticsAnalysis}. 
    \section{数据的缺失值处理}\label{数据的缺失值处理}
        以每日数据为例,我们发现如下问题:
        \begin{enumerate}
            \item [1.] 在疫情发展初期,一些变量有很多缺失值,可以删除行,一些情况下也可以填充为$0$.
            \item [2.] 多个变量体现出了「新增-累计」的特征,如每日新增病例与累计病例、每日新增接种人次与累计接种人次,这些数据会有意外的缺失值,分为以下几种情况:
                \begin{enumerate}
                    \item [(1)] 数值意外地未被记录,但是可以从其周围的数据中恢复;
                    \item [(2)] 某日的新增数据为零,与近几日数据不相符合,可以认为是离群值;这种情况认为是统计滞后引起的,可以通过近几日的平均来进行光滑,也可以不作改动.
                    \item [(3)] 某日的总量数据比前一日乃至前几日的总量数据要少,使得当日的新增数据被记录为缺失值;这种情况认为是统计更正引起的,由于不知道这些多被记录的数据的分布情况,不能准确地作出修改.
                    
                    {\kaishu 我们认为此种情况可以有多种处理方式,这里列举两种:(1-)根据模型的选择,可以不作修改,直接将当日新增数据记录为负值;(2-)选取窗口进行光滑,用近几日数据的平均等统计数字记录当日的新增数据,计算前一日至当日的差值,指数衰减地分配到当日的前几日之前. (之所以使用衰减的分配,我们隐含了一种假设,即数据更正多是发生在案例数快速增长的时期,此时可能因为统计口径的原因多记录了一些案例,这些案例应当更可能分布在时间较近的时刻. 之所以分配到几日之前,我们隐含假设统计和统计更正本身需要时间.)}
                \end{enumerate}
        \end{enumerate}
        为了解决这些问题,我们编写了\texttt{TableMissingValues.m},可以用来分析缺失值的分布,在这之后再经过人工判断,可以选择不同的参数,以恢复数据. 我们也将这个功能整合到了类\texttt{StatisticsAnalysis}中. 下面举一个例子来阐述其用法. 例如对法国在2020年的新增与累计病例数进行缺失值处理,我们只需在上文的基础上——
\begin{spacing}{0.7}
\begin{matlabcode}
daily_SA.MissingValuesReport
\end{matlabcode}
\begin{matlaboutput}
ans = 
        Map: [343x3 logical]
       date: {}
  new_cases: {[72 72] [75 75] [91 91] [97 97] [122 122] [131 132] [157 157] [286 286]}
total_cases: {}
\end{matlaboutput}
\end{spacing}
可见数据中只有新增数据缺失,接着——
\begin{matlabcode}
MissingValuesOptions = { {'new_cases', 'total_cases'}, 'Increment-Addition', ...
        {'InterpolationStyle','LinearRound', ...
        'RemoveFirstRows',false,'RemoveLastRows',false}
    };
daily2 = daily_SA.Update('MissingValuesOptions', MVO).TimeTable;
\end{matlabcode}
这里缺省使用了上文所述的指数衰减,因为其总量数据不单调. 更多的缺失值信息可以通过下面的方法获取——
\begin{spacing}{0.7}
\begin{matlabcode}
daily_SA.MissingValuesReport.increment_addition_new_cases_total_cases
\end{matlabcode}
\begin{matlaboutput}
ans = 
          Increment: 'new_cases'
           Addition: 'total_cases'
     IncrementWhere: [0 1 0]
      AdditionWhere: [0 0 1]
IncrementMissingMap: [343x1 logical]
 AdditionMissingMap: [343x1 logical]
 DecreasingAddition: {[71 72] [74 75] [90 91] [96 97] [121 122] [130 131] 
                      [131 132] [156 157] [285 286]}
      MissingBlocks: [8x1 struct]
    tpMissingBlocks: [8x1 struct]
 MissingBlocksGroup: [8x1 struct]
\end{matlaboutput}
\end{spacing}
另外,自定义的插值、衰减函数可以通过参数用函数句柄导入. 更多的信息请参见附录\ref{app:StatisticsAnalysis}. 
    \section{数据分析过程}\label{数据分析}
    \subsection{一般的统计指标}\label{一般的统计指标}
    \subsection{聚类分析}
    \subsection{疫情数据的分析预测}
    \newpage
    \appendix
    \section{\texttt{StatisticsAnalysis}类}\label{app:StatisticsAnalysis}
      详见github仓库,网址\url{https://github.com/maix00/StatisticsAnalysis}. 本类存储在\texttt{./functions/@StatisticsAnalysis}. 本类目前有如下功能:
    \begin{enumerate}
        \item [1.] \textbf{修改导入参数,并导入表格};
        
        {\kaishu 
        修改导入参数,请使用\texttt{ImportOptions}参数,类型可以是\texttt{struct}或采用name/value-pair的\texttt{cell};name应当是你所要修改的\texttt{detectImportOptions}中得到的属性. 
        
          另外,如果有设置多个导入参数的需要,请将值置于\texttt{1x1 cell},比如导入第$3\sim 10$及$20\sim 30$行,则为\texttt{DataLines:\{\{[3 10] [20 30]\}\}}.
        }
        \item [2.] \textbf{检索表格};
        
        {\kaishu
        传入检索参数,请使用\texttt{SelectTableOptions},类型可以是\texttt{struct}或采用name/value-pair的\texttt{cell};name应当是你所要检索的变量名.

          导入与检索后的表格,可以通过属性名\texttt{Table}或\texttt{TimeTable}访问. 访问原表格请使用属性名\texttt{WholeTable}. 
        }
        \item [3.] \textbf{缺失值探测、修正};
        
        {\kaishu
        传入修正缺失值的参数,请使用\texttt{MissingValuesOptions},类型可以是\texttt{struct}或\texttt{N\_} \texttt{by\_3 cell}. 传入\texttt{cell}时,每行第一位是涉及的变量名\texttt{VariableNames},第二位是使用的修正方法\texttt{Style},第三位是该修正方法所需要的其他参数,类型可以是\texttt{struct}或者采用name/value-pair的\texttt{cell}.

          查看探测信息,请使用属性名\texttt{MissingValuesReport}. 缺失值修正后的表格将覆盖掉\texttt{obj.Table}. 此外,允许的参数值如下表所示:
        \vspace{-2em}
        \begin{flushleft}
        \begin{tabular*}{\textwidth}[H]{c|l|l}
            修正方法&参数名&允许参数值\\ \hline
            \texttt{MissingDetect}&——&——\\ \hline
            \multirow{11}{*}{
                \makecell{\\
                \texttt{Increment-}\\\texttt{Addition}
                }
            }
            &\texttt{RemoveLastRows}
            &缺省\texttt{true}\\ \cline{2-3}
            &\texttt{ConstantValues\_FirstRows}
            &缺省空\\
            &\texttt{RemoveFirstRows}
            &缺省\texttt{true}\\ \cline{2-3}
            &\texttt{InterpolationStyle}
            &\multirow{6}{*}{
                \makecell{
                \texttt{Style}缺省\texttt{"Linear"}或\\\texttt{"LinearRound"},
                允许函数名\\字符串如\texttt{"spline"};\\\texttt{Function}允许函数句柄.
                }
            }\\
            &\texttt{InterpolationFunction}
            &\\
            &\texttt{InterpolationStyle\_P}
            &\\
            &\texttt{InterpolationFunction\_P}
            &\\
            &\texttt{InterpolationStyle\_C}
            &\\
            &\texttt{InterpolationFunction\_C}
            &\\ \cline{2-3}
            &\texttt{DecreasingAdditionStyle}
            &\multirow{2}{*}{
                \makecell{\texttt{Style}缺省\texttt{"Exponential"},允许\\\texttt{"LinearScale"}, \texttt{"DoNothing"}.}
            }\\
            &\texttt{DecreasingAdditionParameters}
            &\\
            \hline
            \multirow{2}{*}{
                \makecell{
                \texttt{Interpolation}
                }
            }
            &\texttt{InterpolationStyle}
            &\\
            &\texttt{InterpolationFunction}
            &\\\hline
            \texttt{ConstantValues}
            &\texttt{ConstantValues}
            &缺省空
        \end{tabular*}
        \end{flushleft}

        其中,函数句柄句法、相关参数有
        \begin{enumerate}
            \item [(1-)]\texttt{T = InterpolationFunction(T, startRow, endRow, VariableMap);}
            \item [(2-)]\texttt{T = InterpolationFunction\_C(T, startRow, endRow, IncrementWhere, Ad-}\\\texttt{ditionWhere);} 注:新增-累计类数据会有两种需要插值的情况,记为$P$与$C$.
            \item [(3-)] 非单调累计数据处理方法\texttt{"Exponential"}的参数较多,包括RoundingWindowAhead, RoundingWindowBehind, RoundingStrategy, RoundingScale, ExponentialRate, AcceptedRatioMinimum, AcceptedRatioMaximum, SpanAheadSkip.
        \end{enumerate}
        }
        
        \item [4.] \textbf{给不同变量贴上不同标签,并对不同标签的变量,计算传入或预设的统计指标,并将计算结果添加到\texttt{Table.Properties.CustomProperties}}.
        
        {\kaishu
        传入标签与统计指标的参数,请使用\texttt{TagsGenerateOptions},类型可以是\texttt{struct}或\texttt{cell};\texttt{struct}的域名,或者\texttt{cell}的左列,应当是函数\texttt{./function/@StatisticsAna-} \texttt{lysis/TagsGenerate}的参数名,包括
        \begin{itemize}[itemsep=-1pt,topsep=1pt]
            \item \texttt{CustomTagName},例如\texttt{\{"continuous", [0 1 1]\}};
            \item \texttt{CustomTagFunction},例如\texttt{\{"continuous", "variance", @(x,y)tsnanvar(x)\}},其中\texttt{x}和\texttt{y}分别是变量所在列全体,及其去重无缺全体.
        \end{itemize} 
        }
    \end{enumerate}

    注:检索功能引用了\texttt{./function/selecttable.m}与MATLAB内置类\texttt{timerange},缺失值修正引用了\texttt{./function/@TableMissingValues}. 更新各种参数可以使用类方法\texttt{Update}.

    \section{利用有关数据可以进行的可视化}
\end{document}